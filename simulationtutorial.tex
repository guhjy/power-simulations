\documentclass[]{article}
\usepackage{lmodern}
\usepackage{amssymb,amsmath}
\usepackage{ifxetex,ifluatex}
\usepackage{fixltx2e} % provides \textsubscript
\ifnum 0\ifxetex 1\fi\ifluatex 1\fi=0 % if pdftex
  \usepackage[T1]{fontenc}
  \usepackage[utf8]{inputenc}
\else % if luatex or xelatex
  \ifxetex
    \usepackage{mathspec}
  \else
    \usepackage{fontspec}
  \fi
  \defaultfontfeatures{Ligatures=TeX,Scale=MatchLowercase}
\fi
% use upquote if available, for straight quotes in verbatim environments
\IfFileExists{upquote.sty}{\usepackage{upquote}}{}
% use microtype if available
\IfFileExists{microtype.sty}{%
\usepackage{microtype}
\UseMicrotypeSet[protrusion]{basicmath} % disable protrusion for tt fonts
}{}
\usepackage[margin=1in]{geometry}
\usepackage{hyperref}
\hypersetup{unicode=true,
            pdftitle={Power analysis for JCAL special issue},
            pdfauthor={Joshua Rosenberg},
            pdfborder={0 0 0},
            breaklinks=true}
\urlstyle{same}  % don't use monospace font for urls
\usepackage{color}
\usepackage{fancyvrb}
\newcommand{\VerbBar}{|}
\newcommand{\VERB}{\Verb[commandchars=\\\{\}]}
\DefineVerbatimEnvironment{Highlighting}{Verbatim}{commandchars=\\\{\}}
% Add ',fontsize=\small' for more characters per line
\usepackage{framed}
\definecolor{shadecolor}{RGB}{248,248,248}
\newenvironment{Shaded}{\begin{snugshade}}{\end{snugshade}}
\newcommand{\KeywordTok}[1]{\textcolor[rgb]{0.13,0.29,0.53}{\textbf{#1}}}
\newcommand{\DataTypeTok}[1]{\textcolor[rgb]{0.13,0.29,0.53}{#1}}
\newcommand{\DecValTok}[1]{\textcolor[rgb]{0.00,0.00,0.81}{#1}}
\newcommand{\BaseNTok}[1]{\textcolor[rgb]{0.00,0.00,0.81}{#1}}
\newcommand{\FloatTok}[1]{\textcolor[rgb]{0.00,0.00,0.81}{#1}}
\newcommand{\ConstantTok}[1]{\textcolor[rgb]{0.00,0.00,0.00}{#1}}
\newcommand{\CharTok}[1]{\textcolor[rgb]{0.31,0.60,0.02}{#1}}
\newcommand{\SpecialCharTok}[1]{\textcolor[rgb]{0.00,0.00,0.00}{#1}}
\newcommand{\StringTok}[1]{\textcolor[rgb]{0.31,0.60,0.02}{#1}}
\newcommand{\VerbatimStringTok}[1]{\textcolor[rgb]{0.31,0.60,0.02}{#1}}
\newcommand{\SpecialStringTok}[1]{\textcolor[rgb]{0.31,0.60,0.02}{#1}}
\newcommand{\ImportTok}[1]{#1}
\newcommand{\CommentTok}[1]{\textcolor[rgb]{0.56,0.35,0.01}{\textit{#1}}}
\newcommand{\DocumentationTok}[1]{\textcolor[rgb]{0.56,0.35,0.01}{\textbf{\textit{#1}}}}
\newcommand{\AnnotationTok}[1]{\textcolor[rgb]{0.56,0.35,0.01}{\textbf{\textit{#1}}}}
\newcommand{\CommentVarTok}[1]{\textcolor[rgb]{0.56,0.35,0.01}{\textbf{\textit{#1}}}}
\newcommand{\OtherTok}[1]{\textcolor[rgb]{0.56,0.35,0.01}{#1}}
\newcommand{\FunctionTok}[1]{\textcolor[rgb]{0.00,0.00,0.00}{#1}}
\newcommand{\VariableTok}[1]{\textcolor[rgb]{0.00,0.00,0.00}{#1}}
\newcommand{\ControlFlowTok}[1]{\textcolor[rgb]{0.13,0.29,0.53}{\textbf{#1}}}
\newcommand{\OperatorTok}[1]{\textcolor[rgb]{0.81,0.36,0.00}{\textbf{#1}}}
\newcommand{\BuiltInTok}[1]{#1}
\newcommand{\ExtensionTok}[1]{#1}
\newcommand{\PreprocessorTok}[1]{\textcolor[rgb]{0.56,0.35,0.01}{\textit{#1}}}
\newcommand{\AttributeTok}[1]{\textcolor[rgb]{0.77,0.63,0.00}{#1}}
\newcommand{\RegionMarkerTok}[1]{#1}
\newcommand{\InformationTok}[1]{\textcolor[rgb]{0.56,0.35,0.01}{\textbf{\textit{#1}}}}
\newcommand{\WarningTok}[1]{\textcolor[rgb]{0.56,0.35,0.01}{\textbf{\textit{#1}}}}
\newcommand{\AlertTok}[1]{\textcolor[rgb]{0.94,0.16,0.16}{#1}}
\newcommand{\ErrorTok}[1]{\textcolor[rgb]{0.64,0.00,0.00}{\textbf{#1}}}
\newcommand{\NormalTok}[1]{#1}
\usepackage{graphicx,grffile}
\makeatletter
\def\maxwidth{\ifdim\Gin@nat@width>\linewidth\linewidth\else\Gin@nat@width\fi}
\def\maxheight{\ifdim\Gin@nat@height>\textheight\textheight\else\Gin@nat@height\fi}
\makeatother
% Scale images if necessary, so that they will not overflow the page
% margins by default, and it is still possible to overwrite the defaults
% using explicit options in \includegraphics[width, height, ...]{}
\setkeys{Gin}{width=\maxwidth,height=\maxheight,keepaspectratio}
\IfFileExists{parskip.sty}{%
\usepackage{parskip}
}{% else
\setlength{\parindent}{0pt}
\setlength{\parskip}{6pt plus 2pt minus 1pt}
}
\setlength{\emergencystretch}{3em}  % prevent overfull lines
\providecommand{\tightlist}{%
  \setlength{\itemsep}{0pt}\setlength{\parskip}{0pt}}
\setcounter{secnumdepth}{0}
% Redefines (sub)paragraphs to behave more like sections
\ifx\paragraph\undefined\else
\let\oldparagraph\paragraph
\renewcommand{\paragraph}[1]{\oldparagraph{#1}\mbox{}}
\fi
\ifx\subparagraph\undefined\else
\let\oldsubparagraph\subparagraph
\renewcommand{\subparagraph}[1]{\oldsubparagraph{#1}\mbox{}}
\fi

%%% Use protect on footnotes to avoid problems with footnotes in titles
\let\rmarkdownfootnote\footnote%
\def\footnote{\protect\rmarkdownfootnote}

%%% Change title format to be more compact
\usepackage{titling}

% Create subtitle command for use in maketitle
\newcommand{\subtitle}[1]{
  \posttitle{
    \begin{center}\large#1\end{center}
    }
}

\setlength{\droptitle}{-2em}

  \title{Power analysis for JCAL special issue}
    \pretitle{\vspace{\droptitle}\centering\huge}
  \posttitle{\par}
    \author{Joshua Rosenberg}
    \preauthor{\centering\large\emph}
  \postauthor{\par}
      \predate{\centering\large\emph}
  \postdate{\par}
    \date{15/11/2018}


\begin{document}
\maketitle

Adapted from Jessie Sun's excellent tutorial.

\subsection{A Priori Power Analysis}\label{a-priori-power-analysis}

Let's get started with an a priori power analysis. What is the smallest
sample size we need to have 80\% power to detect an effect size of
\(\beta_2\) = 0.25, at an alpha level of .05?

First, we need to load the \emph{lavaan} package

\begin{Shaded}
\begin{Highlighting}[]
\KeywordTok{library}\NormalTok{(lavaan)}
\end{Highlighting}
\end{Shaded}

\begin{verbatim}
## This is lavaan 0.6-2
\end{verbatim}

\begin{verbatim}
## lavaan is BETA software! Please report any bugs.
\end{verbatim}

Next, we need to specify the population model, based on the assumptions
in Figure 2, plus our effect size of interest (\(\beta_2\) = 0.25). This
is the model that, at the population level, we assume is generating the
data that we might see in any given dataset.

Basic \emph{lavaan} notation: a double
\textasciitilde{}\textasciitilde{} denotes variances and covariances,
whereas a single \textasciitilde{} denotes a regression path.

\begin{Shaded}
\begin{Highlighting}[]
\NormalTok{popmod1 <-}\StringTok{ '}
\StringTok{# variances are fixed at 1}
\StringTok{x1~~1*x1}
\StringTok{x2~~1*x2}
\StringTok{x3~~1*x3}
\StringTok{x4~~1*x4}
\StringTok{x5~~1*x5}
\StringTok{x6~~1*x6}
\StringTok{x7~~1*x7}
\StringTok{x8~~1*x8}
\StringTok{xx1~~1*xx1}

\StringTok{# correlation between X1 and X2 is assumed to be .30}
\StringTok{# x1~~.3*x2}

\StringTok{# regression path is assumed to be .25}
\StringTok{y~.25*x1}
\StringTok{y~.25*x2}
\StringTok{y~.25*x3}
\StringTok{y~.25*x4}
\StringTok{y~-.25*x5}
\StringTok{y~-.25*x6}
\StringTok{y~-.25*x7}
\StringTok{y~-.25*x8}

\StringTok{x1~.25*xx1}
\StringTok{x2~.25*xx1}
\StringTok{x3~.25*xx1}
\StringTok{x4~.25*xx1}
\StringTok{x5~-.25*xx1}
\StringTok{x6~-.25*xx1}
\StringTok{x7~-.25*xx1}
\StringTok{x8~-.25*xx1}

\StringTok{# # residual variance of Y is 1 - (.1^2 + .2^2) = .95}
\StringTok{# y~~.95*y}
\StringTok{'}
\end{Highlighting}
\end{Shaded}

We also need to create another \emph{lavaan} model, without those
population-level assumptions.

\begin{Shaded}
\begin{Highlighting}[]
\NormalTok{fitmod <-}\StringTok{ '}
\StringTok{# # variances of X1 and X2}
\StringTok{# x1~~x1}
\StringTok{# x2~~x2}
\StringTok{# x3~~x3}
\StringTok{# x4~~x4}
\StringTok{# x5~~x5}
\StringTok{# x6~~x6}
\StringTok{# x7~~x7}
\StringTok{# x8~~x8}

\StringTok{# # correlation between X1 and X2}
\StringTok{# x1~~x2}

\StringTok{# regression path for Y on X1}
\StringTok{y~x1}
\StringTok{y~x2}
\StringTok{y~x3}
\StringTok{y~x4}
\StringTok{y~x5}
\StringTok{y~x6}
\StringTok{y~x7}
\StringTok{y~x8}

\StringTok{# regression path of interest, Y on X2}
\StringTok{x1~xx1}
\StringTok{x2~xx1}
\StringTok{x3~xx1}
\StringTok{x4~xx1}
\StringTok{x5~xx1}
\StringTok{x6~xx1}
\StringTok{x7~xx1}
\StringTok{x8~xx1}


\StringTok{# # residual variance of Y}
\StringTok{# y~~y}
\StringTok{'}
\end{Highlighting}
\end{Shaded}

To see the logic of the simulation process, let's first just simulate
one dataset based on the population model, popmod1.

\begin{Shaded}
\begin{Highlighting}[]
\KeywordTok{set.seed}\NormalTok{(}\DecValTok{20181102}\NormalTok{)  }\CommentTok{# setting a seed for reproducibility of the example}
\NormalTok{data <-}\StringTok{ }\KeywordTok{simulateData}\NormalTok{(popmod1, }\DataTypeTok{sample.nobs =} \DecValTok{300}\NormalTok{)  }\CommentTok{# assume 500 participants for now}
\end{Highlighting}
\end{Shaded}

Now, we're going to fit our model (fitmod) to this dataset.

\begin{Shaded}
\begin{Highlighting}[]
\NormalTok{fit <-}\StringTok{ }\KeywordTok{sem}\NormalTok{(}\DataTypeTok{model =}\NormalTok{ fitmod, }\DataTypeTok{data=}\NormalTok{data, }\DataTypeTok{fixed.x=}\NormalTok{F)}
\end{Highlighting}
\end{Shaded}

Here are the parameter estimates. The parameter of interest, y
\textasciitilde{} x2, is in row 5. As you can see, this parameter was
statistically significant (\emph{p} \textless{} .005) in this simulation
based on one dataset.

\begin{Shaded}
\begin{Highlighting}[]
\KeywordTok{parameterEstimates}\NormalTok{(fit)  }\CommentTok{# see all parameters}
\end{Highlighting}
\end{Shaded}

\begin{verbatim}
##    lhs op rhs    est    se      z pvalue ci.lower ci.upper
## 1    y  ~  x1  0.270 0.049  5.501  0.000    0.174    0.367
## 2    y  ~  x2  0.237 0.047  5.020  0.000    0.145    0.330
## 3    y  ~  x3  0.240 0.050  4.807  0.000    0.142    0.338
## 4    y  ~  x4  0.220 0.046  4.737  0.000    0.129    0.311
## 5    y  ~  x5 -0.246 0.046 -5.310  0.000   -0.337   -0.155
## 6    y  ~  x6 -0.329 0.055 -6.025  0.000   -0.436   -0.222
## 7    y  ~  x7 -0.210 0.050 -4.228  0.000   -0.307   -0.112
## 8    y  ~  x8 -0.288 0.048 -5.958  0.000   -0.383   -0.193
## 9   x1  ~ xx1  0.197 0.060  3.293  0.001    0.080    0.315
## 10  x2  ~ xx1  0.185 0.062  2.968  0.003    0.063    0.308
## 11  x3  ~ xx1  0.242 0.058  4.142  0.000    0.127    0.356
## 12  x4  ~ xx1  0.317 0.062  5.086  0.000    0.195    0.439
## 13  x5  ~ xx1 -0.258 0.063 -4.096  0.000   -0.382   -0.135
## 14  x6  ~ xx1 -0.188 0.054 -3.489  0.000   -0.293   -0.082
## 15  x7  ~ xx1 -0.276 0.059 -4.718  0.000   -0.391   -0.161
## 16  x8  ~ xx1 -0.199 0.061 -3.264  0.001   -0.318   -0.079
## 17   y ~~   y  0.776 0.063 12.247  0.000    0.652    0.900
## 18  x1 ~~  x1  1.044 0.085 12.247  0.000    0.877    1.211
## 19  x2 ~~  x2  1.133 0.092 12.247  0.000    0.952    1.314
## 20  x3 ~~  x3  0.993 0.081 12.247  0.000    0.834    1.152
## 21  x4 ~~  x4  1.129 0.092 12.247  0.000    0.949    1.310
## 22  x5 ~~  x5  1.158 0.095 12.247  0.000    0.973    1.344
## 23  x6 ~~  x6  0.843 0.069 12.247  0.000    0.708    0.978
## 24  x7 ~~  x7  0.997 0.081 12.247  0.000    0.837    1.156
## 25  x8 ~~  x8  1.081 0.088 12.247  0.000    0.908    1.254
## 26 xx1 ~~ xx1  0.970 0.079 12.247  0.000    0.815    1.125
\end{verbatim}

\begin{Shaded}
\begin{Highlighting}[]
\CommentTok{# parameterEstimates(fit)[5,]  # isolating the row with the parameter of interest}
\end{Highlighting}
\end{Shaded}

However, to estimate power, we need to simulate many datasets. Then, we
can obtain the \% of datasets in which the parameter of interest is
statistically significant. This is our power estimate.

So, let's go ahead and simulate 1000 datasets, still assuming a sample
size of 500.

\begin{Shaded}
\begin{Highlighting}[]
\KeywordTok{library}\NormalTok{(tidyverse)}
\end{Highlighting}
\end{Shaded}

\begin{verbatim}
## -- Attaching packages ---------------------------------- tidyverse 1.2.1 --
\end{verbatim}

\begin{verbatim}
## √ ggplot2 3.1.0     √ purrr   0.2.5
## √ tibble  1.4.2     √ dplyr   0.7.7
## √ tidyr   0.8.2     √ stringr 1.3.1
## √ readr   1.1.1     √ forcats 0.3.0
\end{verbatim}

\begin{verbatim}
## -- Conflicts ------------------------------------- tidyverse_conflicts() --
## x dplyr::filter() masks stats::filter()
## x dplyr::lag()    masks stats::lag()
\end{verbatim}

\begin{Shaded}
\begin{Highlighting}[]
\NormalTok{f <-}\StringTok{ }\ControlFlowTok{function}\NormalTok{(sample_size_vector, fit_mod, pop_mod, n_iterations) \{}
    
\NormalTok{    simulate_dataset <-}\StringTok{ }\ControlFlowTok{function}\NormalTok{(iteration, fit_mod, pop_mod, sample_size) \{}
\NormalTok{        data <-}\StringTok{ }\KeywordTok{simulateData}\NormalTok{(pop_mod, }\DataTypeTok{sample.nobs =}\NormalTok{ sample_size)}
\NormalTok{        fit <-}\StringTok{ }\KeywordTok{sem}\NormalTok{(}\DataTypeTok{model =}\NormalTok{ fit_mod, }\DataTypeTok{data=}\NormalTok{data, }\DataTypeTok{fixed.x =}\NormalTok{ F) }
\NormalTok{        d <-}\StringTok{ }\KeywordTok{as.data.frame}\NormalTok{(}\KeywordTok{parameterEstimates}\NormalTok{(fit))[}\DecValTok{1}\OperatorTok{:}\DecValTok{16}\NormalTok{, }\KeywordTok{c}\NormalTok{(}\DecValTok{1}\OperatorTok{:}\DecValTok{7}\NormalTok{)]}
\NormalTok{        d}\OperatorTok{$}\NormalTok{iteration <-}\StringTok{ }\NormalTok{iteration}
\NormalTok{        d}
\NormalTok{    \}}
    
\NormalTok{    l <-}\StringTok{ }\NormalTok{purrr}\OperatorTok{::}\KeywordTok{map_df}\NormalTok{(}\DataTypeTok{.x =} \DecValTok{1}\OperatorTok{:}\NormalTok{n_iterations, }\DataTypeTok{.f =}\NormalTok{ simulate_dataset, }\DataTypeTok{fit_mod =}\NormalTok{ fitmod, }\DataTypeTok{pop_mod=}\NormalTok{popmod1, }\DataTypeTok{sample_size =}\NormalTok{ sample_size_vector)}
    
\NormalTok{    dl <-}\StringTok{ }\NormalTok{l }\OperatorTok\StringTok{ }
\StringTok{        }\KeywordTok{as_tibble}\NormalTok{() }\OperatorTok\StringTok{ }
\StringTok{        }\KeywordTok{unite}\NormalTok{(path, lhs, op, rhs, }\DataTypeTok{sep =} \StringTok{""}\NormalTok{) }\OperatorTok\StringTok{ }
\StringTok{        }\KeywordTok{group_by}\NormalTok{(path) }\OperatorTok\StringTok{ }
\StringTok{        }\KeywordTok{summarize}\NormalTok{(}\DataTypeTok{estimated_power =} \KeywordTok{mean}\NormalTok{(pvalue }\OperatorTok{<}\StringTok{ }\NormalTok{.}\DecValTok{05}\NormalTok{))}
    
\NormalTok{    dl}\OperatorTok{$}\NormalTok{sample_size <-}\StringTok{ }\NormalTok{sample_size_vector}
    
\NormalTok{    dl}
\NormalTok{\}}
\end{Highlighting}
\end{Shaded}

\begin{Shaded}
\begin{Highlighting}[]
\NormalTok{ll <-}\StringTok{ }\NormalTok{purrr}\OperatorTok{::}\KeywordTok{map_df}\NormalTok{(}\DataTypeTok{.x =} \KeywordTok{seq}\NormalTok{(}\DecValTok{50}\NormalTok{, }\DecValTok{400}\NormalTok{, }\DataTypeTok{by =} \DecValTok{25}\NormalTok{), }\DataTypeTok{.f =}\NormalTok{ f, }\DataTypeTok{fit_mod =}\NormalTok{ fitmod, }\DataTypeTok{pop_mod=}\NormalTok{popmod1, }\DataTypeTok{n_iterations =} \DecValTok{1000}\NormalTok{)}
\end{Highlighting}
\end{Shaded}

\begin{Shaded}
\begin{Highlighting}[]
\NormalTok{ll }\OperatorTok\StringTok{ }\KeywordTok{group_by}\NormalTok{(sample_size) }\OperatorTok\StringTok{ }\KeywordTok{summarize}\NormalTok{(}\DataTypeTok{mean_estimate_power =} \KeywordTok{mean}\NormalTok{(estimated_power))}
\end{Highlighting}
\end{Shaded}

\begin{verbatim}
## # A tibble: 15 x 2
##    sample_size mean_estimate_power
##          <dbl>               <dbl>
##  1          50               0.465
##  2          75               0.605
##  3         100               0.722
##  4         125               0.808
##  5         150               0.870
##  6         175               0.916
##  7         200               0.944
##  8         225               0.965
##  9         250               0.977
## 10         275               0.984
## 11         300               0.991
## 12         325               0.995
## 13         350               0.997
## 14         375               0.998
## 15         400               0.999
\end{verbatim}


\end{document}
